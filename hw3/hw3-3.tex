\input{../cs374.tex}
\usepackage{amsmath, verbatim, tikz, float}

\usetikzlibrary{arrows,automata}

\oddsidemargin 0in
\evensidemargin 0in
\textwidth 6.5in
\topmargin -0.5in
\textheight 9.0in
\newcommand{\norm}[1]{\left\lVert #1 \right\rVert}
\begin{document}

\solution{Nikhil Unni (nunni2)}{Homework 3}{Spring 2015}
\pagestyle{myheadings}

\begin{enumerate}
  \setcounter{enumi}{1} 
\item
  \begin{enumerate}
    \item
      The grammar is as follows:
      $$\langle \text{Start} \rangle \longrightarrow 1\langle\text{Anything}\rangle 0| 0\langle\text{Anything}\rangle 1| 1\langle\text{Start}\rangle 1|0 \langle\text{Start}\rangle 0$$
      $$\langle \text{Anything} \rangle \longrightarrow 0\langle\text{Anything}\rangle | 1\langle\text{Anything}\rangle | 0 | 1$$

      The grammar is basically that if the ends of the string are symmetric (both 1 or both 0), then the inside cannot be a palindrome, and it's recursively breaking down the problem. Else, if the ends are different, then the inside can be anything, and accordingly ``Anything'' accepts any string.

    \item
      $$\langle \text{Start} \rangle \longrightarrow XY | \epsilon$$
      $$X \longrightarrow ab | aXb$$
      $$Y \longrightarrow bc | bYc$$

      In the grammar, ``X'' works to add a to the string. For every a, we need to add a b to keep it balanced. Similarly ``Y'' adds c at the end, and proceeds every c with a b, to keep it balanced. This way strings are constructed by specifying an arbitrary number of a's and c's, and the number of b's will update accordingly.
  \end{enumerate}

\end{enumerate}
\end{document}