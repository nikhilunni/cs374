\documentclass[12t,letterpaper]{article}

\newenvironment{proof}{\noindent{\bf Proof:}}{\qed\bigskip}

\newtheorem{theorem}{Theorem}
\newtheorem{corollary}{Corollary}
\newtheorem{lemma}{Lemma} 
\newtheorem{claim}{Claim}
\newtheorem{fact}{Fact}
\newtheorem{definition}{Definition}
\newtheorem{assumption}{Assumption}
\newtheorem{observation}{Observation}
\newtheorem{example}{Example}
\newcommand{\qed}{\rule{7pt}{7pt}}

\newcommand{\assignment}[4]{
\thispagestyle{plain} 
\newpage
\setcounter{page}{1}
\noindent
\begin{center}
\framebox{ \vbox{ \hbox to 6.28in
{\bf CS446: Machine Learning \hfill #1}
\vspace{4mm}
\hbox to 6.28in
{\hspace{2.5in}\large\mbox{#2}}
\vspace{4mm}
\hbox to 6.28in
{{\it Handed Out: #3 \hfill Due: #4}}
}}
\end{center}
}

\newcommand{\solution}[3]{
\thispagestyle{plain} 
\newpage
\setcounter{page}{1}
\noindent
\begin{center}
\framebox{ \vbox{ \hbox to 6.28in
{\bf CS 374 \hfill #3}
\vspace{4mm}
\hbox to 6.28in
{\hspace{2.5in}\large\mbox{#2}}
\vspace{4mm}
\hbox to 6.28in
{#1 \hfill}
}}
\end{center}
\markright{#1}
}

\newenvironment{algorithm}
{\begin{center}
\begin{tabular}{|l|}
\hline
\begin{minipage}{1in}
\begin{tabbing}
\quad\=\qquad\=\qquad\=\qquad\=\qquad\=\qquad\=\qquad\=\kill}
{\end{tabbing}
\end{minipage} \\
\hline
\end{tabular}
\end{center}}

\def\Comment#1{\textsf{\textsl{$\langle\!\langle$#1\/$\rangle\!\rangle$}}}


\usepackage{amsmath, verbatim, tikz, float}

\usetikzlibrary{arrows,automata}

\oddsidemargin 0in
\evensidemargin 0in
\textwidth 6.5in
\topmargin -0.5in
\textheight 9.0in
\newcommand{\norm}[1]{\left\lVert #1 \right\rVert}
\begin{document}

\solution{Nikhil Unni (nunni2)}{Homework 3}{Spring 2015}
\pagestyle{myheadings}

\begin{enumerate}
  \setcounter{enumi}{1} 
\item
  \begin{enumerate}
    \item
      The grammar is as follows:
      $$\langle \text{Start} \rangle \longrightarrow 1\langle\text{Anything}\rangle 0| 0\langle\text{Anything}\rangle 1| 1\langle\text{Start}\rangle 1|0 \langle\text{Start}\rangle 0$$
      $$\langle \text{Anything} \rangle \longrightarrow 0\langle\text{Anything}\rangle | 1\langle\text{Anything}\rangle | 0 | 1$$

      The grammar is basically that if the ends of the string are symmetric (both 1 or both 0), then the inside cannot be a palindrome, and it's recursively breaking down the problem. Else, if the ends are different, then the inside can be anything, and accordingly ``Anything'' accepts any string.

    \item
      $$\langle \text{Start} \rangle \longrightarrow XY | \epsilon$$
      $$X \longrightarrow ab | aXb$$
      $$Y \longrightarrow bc | bYc$$

      In the grammar, ``X'' works to add a to the string. For every a, we need to add a b to keep it balanced. Similarly ``Y'' adds c at the end, and proceeds every c with a b, to keep it balanced. This way strings are constructed by specifying an arbitrary number of a's and c's, and the number of b's will update accordingly.
  \end{enumerate}

\end{enumerate}
\end{document}