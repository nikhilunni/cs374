\input{../cs374.tex}
\usepackage{amsmath, verbatim, tikz, float}

\usetikzlibrary{arrows,automata}

\oddsidemargin 0in
\evensidemargin 0in
\textwidth 6.5in
\topmargin -0.5in
\textheight 9.0in
\newcommand{\norm}[1]{\left\lVert #1 \right\rVert}
\begin{document}

\solution{Nikhil Unni (nunni2)}{Homework 3}{Spring 2015}
\pagestyle{myheadings}

\begin{enumerate}
  \setcounter{enumi}{1} 
\item
  \begin{enumerate}
    \item 
      A simple example will show that the parse tree is ambiguous.\\
      If we wanted ``if condition then if condition then a=1 else a=1'', this could be interpreted as ``if condition then (if condition then (a=1)) else (a=1)'' or ``if condition then (if condition then (a=1) else (a=1))''.

    \item
      This can be remedied pretty easily by just adding an ``endif'' part of the grammar. The new grammar is now:
      $$\text{STMT} \longrightarrow \langle \text{ASSIGN} \rangle | \langle \text{IF-THEN} \rangle | \langle \text{IF-THEN-ELSE} \rangle$$
      $$\text{IF-THEN} \longrightarrow \text{ if condition then } \langle \text{ENDIF-STMT} \rangle$$
      $$\text{ENDIF-STMT} \longrightarrow \langle \text{ASSIGN} \rangle | \langle \text{IF-THEN} \rangle$$
      $$\text{IF-THEN-ELSE} \longrightarrow \text{ if condition then } \langle \text{STMT} \rangle \text{ else } \langle \text{STMT} \rangle$$

      This will right associate the ``else'' of any expression.
  \end{enumerate}
  
  
\end{enumerate}
\end{document}