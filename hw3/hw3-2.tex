\documentclass[12t,letterpaper]{article}

\newenvironment{proof}{\noindent{\bf Proof:}}{\qed\bigskip}

\newtheorem{theorem}{Theorem}
\newtheorem{corollary}{Corollary}
\newtheorem{lemma}{Lemma} 
\newtheorem{claim}{Claim}
\newtheorem{fact}{Fact}
\newtheorem{definition}{Definition}
\newtheorem{assumption}{Assumption}
\newtheorem{observation}{Observation}
\newtheorem{example}{Example}
\newcommand{\qed}{\rule{7pt}{7pt}}

\newcommand{\assignment}[4]{
\thispagestyle{plain} 
\newpage
\setcounter{page}{1}
\noindent
\begin{center}
\framebox{ \vbox{ \hbox to 6.28in
{\bf CS446: Machine Learning \hfill #1}
\vspace{4mm}
\hbox to 6.28in
{\hspace{2.5in}\large\mbox{#2}}
\vspace{4mm}
\hbox to 6.28in
{{\it Handed Out: #3 \hfill Due: #4}}
}}
\end{center}
}

\newcommand{\solution}[3]{
\thispagestyle{plain} 
\newpage
\setcounter{page}{1}
\noindent
\begin{center}
\framebox{ \vbox{ \hbox to 6.28in
{\bf CS 374 \hfill #3}
\vspace{4mm}
\hbox to 6.28in
{\hspace{2.5in}\large\mbox{#2}}
\vspace{4mm}
\hbox to 6.28in
{#1 \hfill}
}}
\end{center}
\markright{#1}
}

\newenvironment{algorithm}
{\begin{center}
\begin{tabular}{|l|}
\hline
\begin{minipage}{1in}
\begin{tabbing}
\quad\=\qquad\=\qquad\=\qquad\=\qquad\=\qquad\=\qquad\=\kill}
{\end{tabbing}
\end{minipage} \\
\hline
\end{tabular}
\end{center}}

\def\Comment#1{\textsf{\textsl{$\langle\!\langle$#1\/$\rangle\!\rangle$}}}


\usepackage{amsmath, verbatim, tikz, float}

\usetikzlibrary{arrows,automata}

\oddsidemargin 0in
\evensidemargin 0in
\textwidth 6.5in
\topmargin -0.5in
\textheight 9.0in
\newcommand{\norm}[1]{\left\lVert #1 \right\rVert}
\begin{document}

\solution{Nikhil Unni (nunni2)}{Homework 3}{Spring 2015}
\pagestyle{myheadings}

\begin{enumerate}
  \setcounter{enumi}{1} 
\item
  \begin{enumerate}
    \item 
      A simple example will show that the parse tree is ambiguous.\\
      If we wanted ``if condition then if condition then a=1 else a=1'', this could be interpreted as ``if condition then (if condition then (a=1)) else (a=1)'' or ``if condition then (if condition then (a=1) else (a=1))''.

    \item
      This can be remedied pretty easily by just adding an ``endif'' part of the grammar. The new grammar is now:
      $$\text{STMT} \longrightarrow \langle \text{ASSIGN} \rangle | \langle \text{IF-THEN} \rangle | \langle \text{IF-THEN-ELSE} \rangle$$
      $$\text{IF-THEN} \longrightarrow \text{ if condition then } \langle \text{ENDIF-STMT} \rangle$$
      $$\text{ENDIF-STMT} \longrightarrow \langle \text{ASSIGN} \rangle | \langle \text{IF-THEN} \rangle$$
      $$\text{IF-THEN-ELSE} \longrightarrow \text{ if condition then } \langle \text{STMT} \rangle \text{ else } \langle \text{STMT} \rangle$$

      This will right associate the ``else'' of any expression.
  \end{enumerate}
  
  
\end{enumerate}
\end{document}