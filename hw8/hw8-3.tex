\documentclass[12t,letterpaper]{article}

\newenvironment{proof}{\noindent{\bf Proof:}}{\qed\bigskip}

\newtheorem{theorem}{Theorem}
\newtheorem{corollary}{Corollary}
\newtheorem{lemma}{Lemma} 
\newtheorem{claim}{Claim}
\newtheorem{fact}{Fact}
\newtheorem{definition}{Definition}
\newtheorem{assumption}{Assumption}
\newtheorem{observation}{Observation}
\newtheorem{example}{Example}
\newcommand{\qed}{\rule{7pt}{7pt}}

\newcommand{\assignment}[4]{
\thispagestyle{plain} 
\newpage
\setcounter{page}{1}
\noindent
\begin{center}
\framebox{ \vbox{ \hbox to 6.28in
{\bf CS446: Machine Learning \hfill #1}
\vspace{4mm}
\hbox to 6.28in
{\hspace{2.5in}\large\mbox{#2}}
\vspace{4mm}
\hbox to 6.28in
{{\it Handed Out: #3 \hfill Due: #4}}
}}
\end{center}
}

\newcommand{\solution}[3]{
\thispagestyle{plain} 
\newpage
\setcounter{page}{1}
\noindent
\begin{center}
\framebox{ \vbox{ \hbox to 6.28in
{\bf CS 374 \hfill #3}
\vspace{4mm}
\hbox to 6.28in
{\hspace{2.5in}\large\mbox{#2}}
\vspace{4mm}
\hbox to 6.28in
{#1 \hfill}
}}
\end{center}
\markright{#1}
}

\newenvironment{algorithm}
{\begin{center}
\begin{tabular}{|l|}
\hline
\begin{minipage}{1in}
\begin{tabbing}
\quad\=\qquad\=\qquad\=\qquad\=\qquad\=\qquad\=\qquad\=\kill}
{\end{tabbing}
\end{minipage} \\
\hline
\end{tabular}
\end{center}}

\def\Comment#1{\textsf{\textsl{$\langle\!\langle$#1\/$\rangle\!\rangle$}}}


\usepackage{amsmath, verbatim, tikz, float, pgfplots, framed}
\usepackage[]{algorithm2e}

\usetikzlibrary{arrows,automata}

\oddsidemargin 0in
\evensidemargin 0in
\textwidth 6.5in
\topmargin -0.5in
\textheight 9.0in
\newcommand{\norm}[1]{\left\lVert #1 \right\rVert}
\begin{document}

\solution{Nikhil Unni (nunni2)}{Homework 8}{Spring 2015}
\pagestyle{myheadings}

\begin{enumerate}
  \setcounter{enumi}{2}
  \item
    \begin{enumerate}
      \item
        Say that we run Kruskal's algorithm and we get the following added edges, in order (as in from least to greatest c(e)), and get the sequence $S = S_1, S_2, \ldots, S_k$. We know for sure that this is a MST.\\\\
        Suppose that this MST is not the spanning tree that minimizes the bottleneck weight. Because the edges are ordered in Kruskal's algorithm, we know for a fact that the bottleneck weight of the MST generated by Kruskal's will be $c(S_k)$. \\\\
        If $c(S_k)$ is not the minimum possible bottleneck weight, then there must be another edge with weight less than $c(S_k)$ that is the true minimum, and we must remove $S_k$ from the sequence altogether. But if this was the case, Kruskal's algorithm would have included such an edge, as it came before $S_k$ in the order of sorted edges. The fact that it passed such an edge would mean that it would have introduced a cycle, and so it cannot possibly be part of the MST.\\\\
        Therefore we have a contradiction, and so we know that the MST minimizes the bottleneck weight.

      \item
        From:\footnote{http://people.scs.carleton.ca/~maheshwa/courses/5703COMP/14Seminars/BST-Report.pdf}
        \begin{framed}
          \begin{algorithm}[H]      
            Median $\gets$ CalculateMedian\;
            Partition graph into two subgraphs, grouping all edges $\leq$ Median in one\\
            group, and all edges $>$ Median in the other\;
            
            \If{the smaller grouping is connected}{
              return MinimumBottleneck(smaller subgraph)
            }
            \Else{
              \ForEach{component $\in$ smaller grouping}{
                Combine all vertices of the component into a single node\\                
              }
              return MinimumBottleneck(larger subgraph)\;
            }
            \caption{MinimumBottleneck(V,E)}
          \end{algorithm}
        \end{framed}

        Because, at each stage, we are reducing the problem size by $\frac{1}{2}$, the algorithm runs in O(n) + O$(\frac{1}{2})$ + O$(\frac{1}{4})$ + $\ldots$ = O(n). This is assuming that our median algorithm is linear time in the input size, which is true in the case of median of medians.
    \end{enumerate}

  
\end{enumerate}
\end{document}