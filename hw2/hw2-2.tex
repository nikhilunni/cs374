\documentclass[12t,letterpaper]{article}

\newenvironment{proof}{\noindent{\bf Proof:}}{\qed\bigskip}

\newtheorem{theorem}{Theorem}
\newtheorem{corollary}{Corollary}
\newtheorem{lemma}{Lemma} 
\newtheorem{claim}{Claim}
\newtheorem{fact}{Fact}
\newtheorem{definition}{Definition}
\newtheorem{assumption}{Assumption}
\newtheorem{observation}{Observation}
\newtheorem{example}{Example}
\newcommand{\qed}{\rule{7pt}{7pt}}

\newcommand{\assignment}[4]{
\thispagestyle{plain} 
\newpage
\setcounter{page}{1}
\noindent
\begin{center}
\framebox{ \vbox{ \hbox to 6.28in
{\bf CS446: Machine Learning \hfill #1}
\vspace{4mm}
\hbox to 6.28in
{\hspace{2.5in}\large\mbox{#2}}
\vspace{4mm}
\hbox to 6.28in
{{\it Handed Out: #3 \hfill Due: #4}}
}}
\end{center}
}

\newcommand{\solution}[3]{
\thispagestyle{plain} 
\newpage
\setcounter{page}{1}
\noindent
\begin{center}
\framebox{ \vbox{ \hbox to 6.28in
{\bf CS 374 \hfill #3}
\vspace{4mm}
\hbox to 6.28in
{\hspace{2.5in}\large\mbox{#2}}
\vspace{4mm}
\hbox to 6.28in
{#1 \hfill}
}}
\end{center}
\markright{#1}
}

\newenvironment{algorithm}
{\begin{center}
\begin{tabular}{|l|}
\hline
\begin{minipage}{1in}
\begin{tabbing}
\quad\=\qquad\=\qquad\=\qquad\=\qquad\=\qquad\=\qquad\=\kill}
{\end{tabbing}
\end{minipage} \\
\hline
\end{tabular}
\end{center}}

\def\Comment#1{\textsf{\textsl{$\langle\!\langle$#1\/$\rangle\!\rangle$}}}


\usepackage{amsmath, verbatim, tikz, float}

\usetikzlibrary{arrows,automata}

\oddsidemargin 0in
\evensidemargin 0in
\textwidth 6.5in
\topmargin -0.5in
\textheight 9.0in
\newcommand{\norm}[1]{\left\lVert #1 \right\rVert}
\begin{document}

\solution{Nikhil Unni (nunni2)}{Homework 2}{Spring 2015}
\pagestyle{myheadings}

\begin{enumerate}
  \setcounter{enumi}{1}
\item
  First, to do this problem we must define a new $\delta$ function, which we'll call $\delta_2$. It is to represent the state transitions for $L^R$, and it's defined as : $\delta_2(p,a) = \{q | \delta(q,a) = p, q \in Q\}$. Then our new NFA is defined as follows:
  $$Q^N = Q \times Q \cup q_0^N$$
  $$\Sigma^N = \Sigma$$
  $$\delta^N(q_0^N, \epsilon) = \{(q_0, r), r \in F\}$$
  $$\delta^N((p,q), a) = (\delta(p,a), \delta_2(q,a))$$
  $$q_0^N$$
  $$F^N = \{(p,q) | \delta(p,a) = q, \delta_2(q,a) = p, p \in Q, q \in Q\, \text{ for some } a \in \Sigma\}$$

  Essentially all the NFA is doing is that it has several ``starting'' points, where it marks two indices: one is the index of $q_0$, and the other is an element from $F$, and the idea is that the two will walk towards each other.\\\\
  To actually implement multiple possible starting points \footnote{https://piazza.com/class/i4mrvddxr0h3sd?cid=265}, I just have an artificial $q_0^N$ that $\epsilon$ transitions to all of the starting points that I want.\\\\
  The two walk towards each other, with the first index incrementing with the forwards $\delta$, and the second index decrementing with the backward $\delta_2$. Eventually they reach a point where they're one transition away from one another, at the last character of $w$. At this point, we see that the set of final states has this case included, and we can return successfully.\\
  

\end{enumerate}
\end{document}