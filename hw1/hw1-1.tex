\input{../cs374.tex}
\usepackage{amsmath}

\oddsidemargin 0in
\evensidemargin 0in
\textwidth 6.5in
\topmargin -0.5in
\textheight 9.0in
\newcommand{\norm}[1]{\left\lVert #1 \right\rVert}
\begin{document}

\solution{Nikhil Unni (nunni2)}{Homework 1}{Spring 2015}
\pagestyle{myheadings}

\begin{enumerate}
\item
  I'll be inducting on the number of operations to construct a regular language. \footnote{Creds to Prof. Pitt : https://piazza.com/class/i4mrvddxr0h3sd?cid=185}\\\\

  Base Case:\\
  The base case is when the number of operations to construct the regular language is $n = 1$. The only way to get a regular language of a single operation is through the base case of the inductive definition of regular languages:
  \begin{itemize}
  \item $\emptyset$ is represented by a top-plus regular expression, of the form ($\alpha_1 + \ldots + \alpha_k$), where k = 1, because $\alpha_1 = \emptyset$ contains no `+'
  \item $\{\epsilon\}$, is also represented by a top-plus expression, again where k = 1, where $\alpha_1 = \epsilon$ contains no `+'.
  \item $\{a\}$, for any $a$ in any arbitrary alphabet, $\Sigma$, is represented by a top-plus expression, where k = 1, where $\alpha_1 = a$ contains no `+'
  \end{itemize}
  Inductive Hypothesis:\\
  Now assume all $r_k$ and $r_l$ where $1 \leq k < n$, $1 \leq l < m$, where $r_k$ and $r_l$ are regular expressions that represent regular languages, can be represented by a top-plus regular expression.\\\\
  Inductive Step:\\
  Without loss of generality, let $r_{n-1}$ be any regular expression (representing a regular language) that was constructed in $(n-1)$ steps, and let $r_{m-1}$ be any regular expression constructed in $(m-1)$ steps. Since they are both top-plus expressions, say that $r_{n-1}$ is a top-plus expression with $k=x$ terms, and let $r_{m-1}$ be a top-plus expression with $k=y$ terms.\\
  We can get a top-plus regular expression of construction size $n$ by constructing on $r_{n-1}$:
  \begin{itemize}
  \item $(r_{n-1} + r_{m-1})$ is a top-plus expression because it is the union of $x$ and $y$ terms all without any `+'. The result is a top-plus expression of $k=x+y$.
  \item $(r_{n-1}r_{m-1})$ is a top-plus expression as well. Through the distributive law of concatenation over union, we have $(r_{n-1}^1 + r_{n-1}^2 + \ldots + r_{n-1}^x)(r_{m-1}^1 + r_{m-1}^2 + \ldots + r_{m-1}^y) = (r_{n-1}^1r_{m-1}^1 + r_{n-1}^1r_{m-1}^2 + \ldots r_{n-1}^xr_{m-1}^y)$, which represents a top-plus expression of $k=xy$. (Here the ``exponents'' represent indices.)
  \item $(r_{n-1})^*$ is also a top-plus expression. Through the theorem included with the pset, we know that $(r_{n-1}^1+r_{n-1}^2+\ldots+r_{n-1}^x)^* = ({r_{n-1}^1}^{*}{r_{n-1}^2}^{*}\ldots {r_{n-1}^x}^{*})$, which is a top-plus expression of $k=1$.    
  \end{itemize}
  Thus, every regular language can be represented by a top-plus regular expression.  

\end{enumerate}
\end{document}