\documentclass[12t,letterpaper]{article}

\newenvironment{proof}{\noindent{\bf Proof:}}{\qed\bigskip}

\newtheorem{theorem}{Theorem}
\newtheorem{corollary}{Corollary}
\newtheorem{lemma}{Lemma} 
\newtheorem{claim}{Claim}
\newtheorem{fact}{Fact}
\newtheorem{definition}{Definition}
\newtheorem{assumption}{Assumption}
\newtheorem{observation}{Observation}
\newtheorem{example}{Example}
\newcommand{\qed}{\rule{7pt}{7pt}}

\newcommand{\assignment}[4]{
\thispagestyle{plain} 
\newpage
\setcounter{page}{1}
\noindent
\begin{center}
\framebox{ \vbox{ \hbox to 6.28in
{\bf CS446: Machine Learning \hfill #1}
\vspace{4mm}
\hbox to 6.28in
{\hspace{2.5in}\large\mbox{#2}}
\vspace{4mm}
\hbox to 6.28in
{{\it Handed Out: #3 \hfill Due: #4}}
}}
\end{center}
}

\newcommand{\solution}[3]{
\thispagestyle{plain} 
\newpage
\setcounter{page}{1}
\noindent
\begin{center}
\framebox{ \vbox{ \hbox to 6.28in
{\bf CS 374 \hfill #3}
\vspace{4mm}
\hbox to 6.28in
{\hspace{2.5in}\large\mbox{#2}}
\vspace{4mm}
\hbox to 6.28in
{#1 \hfill}
}}
\end{center}
\markright{#1}
}

\newenvironment{algorithm}
{\begin{center}
\begin{tabular}{|l|}
\hline
\begin{minipage}{1in}
\begin{tabbing}
\quad\=\qquad\=\qquad\=\qquad\=\qquad\=\qquad\=\qquad\=\kill}
{\end{tabbing}
\end{minipage} \\
\hline
\end{tabular}
\end{center}}

\def\Comment#1{\textsf{\textsl{$\langle\!\langle$#1\/$\rangle\!\rangle$}}}


\usepackage{amsmath}

\oddsidemargin 0in
\evensidemargin 0in
\textwidth 6.5in
\topmargin -0.5in
\textheight 9.0in
\newcommand{\norm}[1]{\left\lVert #1 \right\rVert}
\begin{document}

\solution{Nikhil Unni (nunni2)}{Homework 1}{Spring 2015}
\pagestyle{myheadings}

\begin{enumerate}
\item
  I'll be inducting on the number of operations to construct a regular language.\\\\

  Base Case:\\
  The base case is when the number of operations to construct the regular language is $n = 1$. The only way to get a regular language of a single operation is through the base case of the inductive definition of regular languages:
  \begin{itemize}
  \item $\emptyset$ is represented by a top-plus regular expression, of the form ($\alpha_1 + \ldots + \alpha_k$), where k = 1, because $\alpha_1 = \emptyset$ contains no `+'
  \item $\{\epsilon\}$, is also represented by a top-plus expression, again where k = 1, where $\alpha_1 = \epsilon$ contains no `+'.
  \item $\{a\}$, for any $a$ in any arbitrary alphabet, $\Sigma$, is represented by a top-plus expression, where k = 1, where $\alpha_1 = a$ contains no `+'
  \end{itemize}
  Inductive Hypothesis:\\
  Now assume for all regular languages, where the number of operations to construct is $1 \leq k < n$, that they can be represented by a top-plus regular expression.\\\\
  Inductive Step:\\
  Without loss of generality, let $r_{n-1}$ be any regular expression (representing a regular language) that was constructed in $(n-1)$ steps.
  

\end{enumerate}
\end{document}