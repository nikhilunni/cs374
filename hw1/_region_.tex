\message{ !name(hw1-3.tex)}\documentclass[12t,letterpaper]{article}

\newenvironment{proof}{\noindent{\bf Proof:}}{\qed\bigskip}

\newtheorem{theorem}{Theorem}
\newtheorem{corollary}{Corollary}
\newtheorem{lemma}{Lemma} 
\newtheorem{claim}{Claim}
\newtheorem{fact}{Fact}
\newtheorem{definition}{Definition}
\newtheorem{assumption}{Assumption}
\newtheorem{observation}{Observation}
\newtheorem{example}{Example}
\newcommand{\qed}{\rule{7pt}{7pt}}

\newcommand{\assignment}[4]{
\thispagestyle{plain} 
\newpage
\setcounter{page}{1}
\noindent
\begin{center}
\framebox{ \vbox{ \hbox to 6.28in
{\bf CS446: Machine Learning \hfill #1}
\vspace{4mm}
\hbox to 6.28in
{\hspace{2.5in}\large\mbox{#2}}
\vspace{4mm}
\hbox to 6.28in
{{\it Handed Out: #3 \hfill Due: #4}}
}}
\end{center}
}

\newcommand{\solution}[3]{
\thispagestyle{plain} 
\newpage
\setcounter{page}{1}
\noindent
\begin{center}
\framebox{ \vbox{ \hbox to 6.28in
{\bf CS 374 \hfill #3}
\vspace{4mm}
\hbox to 6.28in
{\hspace{2.5in}\large\mbox{#2}}
\vspace{4mm}
\hbox to 6.28in
{#1 \hfill}
}}
\end{center}
\markright{#1}
}

\newenvironment{algorithm}
{\begin{center}
\begin{tabular}{|l|}
\hline
\begin{minipage}{1in}
\begin{tabbing}
\quad\=\qquad\=\qquad\=\qquad\=\qquad\=\qquad\=\qquad\=\kill}
{\end{tabbing}
\end{minipage} \\
\hline
\end{tabular}
\end{center}}

\def\Comment#1{\textsf{\textsl{$\langle\!\langle$#1\/$\rangle\!\rangle$}}}


\usepackage{amsmath}

\oddsidemargin 0in
\evensidemargin 0in
\textwidth 6.5in
\topmargin -0.5in
\textheight 9.0in
\newcommand{\norm}[1]{\left\lVert #1 \right\rVert}
\begin{document}

\message{ !name(hw1-3.tex) !offset(-3) }


\solution{Nikhil Unni (nunni2)}{Homework 1}{Spring 2015}
\pagestyle{myheadings}

\begin{enumerate}
  \setcounter{enumi}{2}
\item
  \begin{enumerate}
    \item
      $$Q := Q_1 \times Q_2 \times Q_3$$
      $$\Sigma := \Sigma$$
      $$\delta ((p,q,r), a) := (\delta_1(p,a), \delta_2(q,a), \delta_3(r,a)), p \in Q_1, q \in Q_2, r \in Q_3, a \in \Sigma$$
      $$q_0 := (q_0^{(1)}, q_0^{(2)}, q_0^{(3)})$$
      $$F := (F_1 \times F_2 \times Q_3) \cup (F_1 \times Q_2 \times F_3) \cup (Q_1 \times F_2 \times F_3)$$
      Essentially, the accepting states are if you accept any two DFAs. So I just took all $3 \choose 2$ ways to succeed. The case where all 3 DFAs are fulfilled is included implicitly.\\

    \item
      First we have to prove that the extended $\delta^*$ function is accurate for all words with induction. \footnote{Proof outline inspired by past lectures : https://courses.engr.illinois.edu/cs373/fa2011/lectures/lect\_04.pdf} \\
      So we're trying to prove that $\delta^*((q_0^{(1)},q_0^{(2)},q_0^{(3)}), w) = (\delta_1^*(q_0^{(1)}, w),\delta_2^*(q_0^{(2)}, w), \delta_3^*(q_0^{(3)}, w)), w \in \Sigma^*$ with induction on the length of the input:\\\\

      Base Case:\\
      The length of the string is 0, or is $\epsilon$.\\
      $$\delta_1^*(q_0^{(1)}, \epsilon) = q_0^{(1)}, \delta_2^*(q_0^{(2)}, \epsilon) = q_0^{(2)}, \delta_3^*(q_0^{(3)}, \epsilon) = q_0^{(3)}$$
      $\epsilon$ as the input for our new $\delta^*$ function just returns the start state. So we get:
      $$\delta^*((q_0^{(1)},q_0^{(2)},q_0^{(3)}), \epsilon) = (q_0^{(1)},q_0^{(2)},q_0^{(3)}) = (\delta_1^*(q_0^{(1)}, \epsilon),\delta_2^*(q_0^{(2)}, \epsilon), \delta_3^*(q_0^{(3)}, \epsilon))$$

      Inductive Hypothesis:\\
      Assume that for all words of length k, such that $0 \leq k < n$, that $\delta^* ((p,q,r), w_k) = (\delta_1^*(p,w_k), \delta_2^*(q,w_k), \delta_3^*(r,w_k))$\\\\

      Inductive Step:\\
      Let $w_n = wa$ be a word of length n, such that $\|a\| = 1, \|w\| = n-1$. Then let the state of $Q$ after reading $w$ be $(l,m,n)$.\\
      This also means that $\delta_1^*(q_0^{(1)}, w) = l, \delta_2^*(q_0^{(2)}, w) = m, \delta_3^*(q_0^{(3)}, w) = n$ by the inductive hypothesis.\\
      Now let $\delta((l,m,n), a) = (x,y,z)$. From the original definition of our delta function, we know that $\delta_1(l,a) = x, \delta_2(m,a) = y, \delta_1(n,a) = z$.\\\\
      So by the inductive assumption, we know that $\delta^*((q_0^{(1)},q_0^{(2)},q_0^{(3)}), w) = (\delta_1^*(q_0^{(1)}, w),\delta_2^*(q_0^{(2)}, w), \delta_3^*(q_0^{(3)}, w))$\\\\

      We now want to show that $L(Q) = (L(Q_1) \cap L(Q_2)) \cup (L(Q_1) \cap L(Q_3)) \cup (L(Q_2) \cap L(Q_3))$\\
      Let : $w \in L(Q) = (L(Q_1) \cap L(Q_2)) \cup (L(Q_1) \cap L(Q_3)) \cup (L(Q_2) \cap L(Q_3))$. \\
      Let there be some final states x,y,z, s.t. $x = \delta_1^*(q_0^{(1)},w), y = \delta_2^*(q_0^{(2)},w), z = \delta_3^*(q_0^{(3)},w)$, such that at least 2 of x,y, or z are in $F_1, F_2, F_3$ respectively.\\
      From the definition of our new F, w is in the language $L(Q)$, so $(L(Q_1) \cap L(Q_2)) \cup (L(Q_1) \cap L(Q_3)) \cup (L(Q_2) \cap L(Q_3)) \in L(Q)$.\\\\

      Similarly, we can show that if w is an element of $L(Q)$, it must be a member of the set $(L(Q_1) \cap L(Q_2)) \cup (L(Q_1) \cap L(Q_3)) \cup (L(Q_2) \cap L(Q_3))$, showing that the two sets must be equal.
      
  \end{enumerate}

\end{enumerate}
\end{document}
\message{ !name(hw1-3.tex) !offset(-60) }
