\documentclass[12t,letterpaper]{article}

\newenvironment{proof}{\noindent{\bf Proof:}}{\qed\bigskip}

\newtheorem{theorem}{Theorem}
\newtheorem{corollary}{Corollary}
\newtheorem{lemma}{Lemma} 
\newtheorem{claim}{Claim}
\newtheorem{fact}{Fact}
\newtheorem{definition}{Definition}
\newtheorem{assumption}{Assumption}
\newtheorem{observation}{Observation}
\newtheorem{example}{Example}
\newcommand{\qed}{\rule{7pt}{7pt}}

\newcommand{\assignment}[4]{
\thispagestyle{plain} 
\newpage
\setcounter{page}{1}
\noindent
\begin{center}
\framebox{ \vbox{ \hbox to 6.28in
{\bf CS446: Machine Learning \hfill #1}
\vspace{4mm}
\hbox to 6.28in
{\hspace{2.5in}\large\mbox{#2}}
\vspace{4mm}
\hbox to 6.28in
{{\it Handed Out: #3 \hfill Due: #4}}
}}
\end{center}
}

\newcommand{\solution}[3]{
\thispagestyle{plain} 
\newpage
\setcounter{page}{1}
\noindent
\begin{center}
\framebox{ \vbox{ \hbox to 6.28in
{\bf CS 374 \hfill #3}
\vspace{4mm}
\hbox to 6.28in
{\hspace{2.5in}\large\mbox{#2}}
\vspace{4mm}
\hbox to 6.28in
{#1 \hfill}
}}
\end{center}
\markright{#1}
}

\newenvironment{algorithm}
{\begin{center}
\begin{tabular}{|l|}
\hline
\begin{minipage}{1in}
\begin{tabbing}
\quad\=\qquad\=\qquad\=\qquad\=\qquad\=\qquad\=\qquad\=\kill}
{\end{tabbing}
\end{minipage} \\
\hline
\end{tabular}
\end{center}}

\def\Comment#1{\textsf{\textsl{$\langle\!\langle$#1\/$\rangle\!\rangle$}}}


\usepackage{amsmath, verbatim, tikz, float, pgfplots, framed}
\usepackage[]{algorithm2e}

\usetikzlibrary{arrows,automata}

\oddsidemargin 0in
\evensidemargin 0in
\textwidth 6.5in
\topmargin -0.5in
\textheight 9.0in
\newcommand{\norm}[1]{\left\lVert #1 \right\rVert}
\begin{document}

\solution{Nikhil Unni (nunni2)}{Homework 6}{Spring 2015}
\pagestyle{myheadings}

\begin{enumerate}
  \setcounter{enumi}{2}
\item
  $$\text{KSHOT}(p, n, k) = K(p, 1, n, k, \emptyset)\\$$
  \begin{displaymath}
    K(p, i, j, k, \text{strat}) = \left\{
      \begin{array}{lr}
        0 & \text{ if } k = 0\\
        0 & \text{ if } j \leq i\\
        \text{max} \left\{ 
          \begin{array}{lr}
            \text{max}_{i+1 \leq x \leq j}(p[x] - p[i] + K(p, x+1, j, k-1, \text{strat} \cup (i,x)))\\
            K(p, i+1, j, k, \text{strat})
          \end{array}
        \right.
      \end{array}
    \right. 
  \end{displaymath}

  \begin{framed}
    \begin{algorithm}[H]
      \If{$k = 0 \text{ or } j \leq i$} {
        return 0\;
      }
      hi $\gets -\infty$\;
      \For{$i+1 \leq x \leq j$} {
        temp $\gets p[x] - p[i] + K(x+1, j, k-1)$\;
        \If{temp $>$ hi} {
          hi = temp\;
        }
      }
      temp $\gets K(i+1, j, k)$\;
      \If{temp $>$ hi} {
        hi = temp\;
      }
      return hi\;
      \caption{K-shot strategies}
    \end{algorithm}
  \end{framed}

  So the algorithm starts out with two indices, i and j, pointing to 1 and n respectively, to encompass the entire initial problem. You can see the correct initial call for the problem, at the top : KSHOT(p,n,k) = $K(p,1,n,k, \emptyset)$. At every stage, the algorithm has a choice to include i in the list of pairs, or move on, where the second index, x, is anything between i+1 and j, inclusive -- in this case, the method is recursively called with the correct indices, a value of k-1, and the strategy so far is the same thing appended with this new $(b_i,s_i)$ tuple.\\

  Or, the algorithm can move on an index, by skipping the current i, and moving on to i+1, with the rest of the parameters the same.\\

  If we've already exhausted the total number of strategies so far, or if we've decremented k, k times (or if k == 0), or if we've reached the bound of our indices i and j, then the algorithm returns 0 to show that it is nonoptimal (yet still better than a negative choice).\\

  I've written out the pseudocode for the same function, although it only calculates the value of the winning k-shot (off by a factor of 1000 from the real value) merely to show the evaluation order. To actually get the set of k-shots, keep track of the set with each recursive call, like in the mathematical formalization above.\\

  As for the actual memoization, there are only $O(kn^2)$ unique problems, and they're all of the substrings of the array p, given the current number of k-shots we have left. We can see this by the indices i,j, and k. The memoization structure woud be a 3D array of size nxnxk, where each position represents the value of K(i,j,k), as the max value of k-shots between a given i and j are also dictated by how many k-shots we are permitted at most. Because each problem takes O(1) time to evaluate, just simple addition and subtraction, in total, the time complexity of the algorithm is $O(kn^2)$.
  

\end{enumerate}
\end{document}