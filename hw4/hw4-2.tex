\input{../cs374.tex}
\usepackage{amsmath, verbatim, tikz, float}

\usetikzlibrary{arrows,automata}

\oddsidemargin 0in
\evensidemargin 0in
\textwidth 6.5in
\topmargin -0.5in
\textheight 9.0in
\newcommand{\norm}[1]{\left\lVert #1 \right\rVert}
\begin{document}

\solution{Nikhil Unni (nunni2)}{Homework 4}{Spring 2015}
\pagestyle{myheadings}

\begin{enumerate}
  \setcounter{enumi}{1} 
\item
  \begin{algorithm}
    DangerousWalk(G):\\
    \> BFS from s, and make $A = A \cap $ reach(s), $B = B \cap$ reach(s)\\
    \>Mark all nodes in G as unvisited\\
    \>for each u $\in$ A\\
    \>\>if DangerousWalk(u) is dangerous\\
    \>\>\> output ``DANGEROUS''\\
    \>output ``Safe''\\\\\\

    DangerousWalk(u):\\
    \>Mark u\\
    \>For each edge (u,v) in Out(u)\\
    \>\>If v $\in$ A or v $\in$ (V-A-B)\\
    \>\>\>If v is not marked\\
    \>\>\>\>DangerousWalk(v)\\
    \>\>\>else\\
    \>\>\>\>output dangerous\\
    \>output safe\\
  \end{algorithm}
  
  The algorithm hinges on the fact that if there is a dangerous walk, it has to be a cycle of only dangerous or normal vertices. So the algorithm is just a DFS starting from each dangerous node, only taking normal or dangerous vertices as next steps. Once we encounter a cycle, we've successfully found a risky path.
\end{enumerate}
\end{document}