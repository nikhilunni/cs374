\documentclass[12t,letterpaper]{article}

\newenvironment{proof}{\noindent{\bf Proof:}}{\qed\bigskip}

\newtheorem{theorem}{Theorem}
\newtheorem{corollary}{Corollary}
\newtheorem{lemma}{Lemma} 
\newtheorem{claim}{Claim}
\newtheorem{fact}{Fact}
\newtheorem{definition}{Definition}
\newtheorem{assumption}{Assumption}
\newtheorem{observation}{Observation}
\newtheorem{example}{Example}
\newcommand{\qed}{\rule{7pt}{7pt}}

\newcommand{\assignment}[4]{
\thispagestyle{plain} 
\newpage
\setcounter{page}{1}
\noindent
\begin{center}
\framebox{ \vbox{ \hbox to 6.28in
{\bf CS446: Machine Learning \hfill #1}
\vspace{4mm}
\hbox to 6.28in
{\hspace{2.5in}\large\mbox{#2}}
\vspace{4mm}
\hbox to 6.28in
{{\it Handed Out: #3 \hfill Due: #4}}
}}
\end{center}
}

\newcommand{\solution}[3]{
\thispagestyle{plain} 
\newpage
\setcounter{page}{1}
\noindent
\begin{center}
\framebox{ \vbox{ \hbox to 6.28in
{\bf CS 374 \hfill #3}
\vspace{4mm}
\hbox to 6.28in
{\hspace{2.5in}\large\mbox{#2}}
\vspace{4mm}
\hbox to 6.28in
{#1 \hfill}
}}
\end{center}
\markright{#1}
}

\newenvironment{algorithm}
{\begin{center}
\begin{tabular}{|l|}
\hline
\begin{minipage}{1in}
\begin{tabbing}
\quad\=\qquad\=\qquad\=\qquad\=\qquad\=\qquad\=\qquad\=\kill}
{\end{tabbing}
\end{minipage} \\
\hline
\end{tabular}
\end{center}}

\def\Comment#1{\textsf{\textsl{$\langle\!\langle$#1\/$\rangle\!\rangle$}}}


\usepackage{amsmath, verbatim, tikz, float}

\usetikzlibrary{arrows,automata}

\oddsidemargin 0in
\evensidemargin 0in
\textwidth 6.5in
\topmargin -0.5in
\textheight 9.0in
\newcommand{\norm}[1]{\left\lVert #1 \right\rVert}
\begin{document}

\solution{Nikhil Unni (nunni2)}{Homework 4}{Spring 2015}
\pagestyle{myheadings}

\begin{enumerate}
  \setcounter{enumi}{1} 
\item
  \begin{algorithm}
    DangerousWalk(G):\\
    \> BFS from s, and make $A = A \cap $ reach(s), $B = B \cap$ reach(s)\\
    \>Mark all nodes in G as unvisited\\
    \>for each u $\in$ A\\
    \>\>if DangerousWalk(u) is dangerous\\
    \>\>\> output ``DANGEROUS''\\
    \>output ``Safe''\\\\\\

    DangerousWalk(u):\\
    \>Mark u\\
    \>For each edge (u,v) in Out(u)\\
    \>\>If v $\in$ A or v $\in$ (V-A-B)\\
    \>\>\>If v is not marked\\
    \>\>\>\>DangerousWalk(v)\\
    \>\>\>else\\
    \>\>\>\>output dangerous\\
    \>output safe\\
  \end{algorithm}
  
  The algorithm hinges on the fact that if there is a dangerous walk, it has to be a cycle of only dangerous or normal vertices. So the algorithm is just a DFS starting from each dangerous node, only taking normal or dangerous vertices as next steps. Once we encounter a cycle, we've successfully found a risky path.
\end{enumerate}
\end{document}